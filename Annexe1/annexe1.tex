\setcounter{figure}{0} 
\setcounter{table}{0}
\setcounter{footnote}{0}
\setcounter{equation}{0}
\pagestyle{fancy}
\fancyhf{}
\renewcommand{\chaptermark}[1]{\markboth{\MakeUppercase{#1 }}{}}
\renewcommand{\sectionmark}[1]{\markright{\thesection~ #1}}
\fancyhead[RO]{\bfseries\rightmark}
\fancyhead[LE]{\bfseries\leftmark}
\fancyfoot[RO]{\thepage}
\fancyfoot[LE]{\thepage}
\renewcommand{\headrulewidth}{0.5pt}
\renewcommand{\footrulewidth}{0pt}

\makeatletter
\renewcommand\thefigure{A.\arabic{figure}}
\renewcommand\thetable{A.\arabic{table}} 
\makeatother

\chapter{Annexe A : Solution Shuttle}
\graphicspath{{Annexe1/figures/}}
%==========================================================================

%    Annexe

%===========================================================================
\begin{figure}[!ht]\centering
\includegraphics[scale=0.31]{"shuttle archi app".png}
\caption{Architecture applicative de la solution Shuttle}
\label{fig:fig1}
\end{figure}
\FloatBarrier
\begin{figure}[!ht]\centering
\includegraphics[scale=0.4]{"annexea2".png}
\caption{Briques technologiques intégrées au sein de la solution Shuttle}
\label{fig:fig1}
\end{figure}
\FloatBarrier
\vspace{-3mm}
\begin{itemize}
\item[A.3.] OLAP :
    \begin{itemize}
    \item[--] Vision conceptuelle multidimensionnelle.
    \item[--] Indexation de la donnée.
    \item[--] Calculs analytique rapides et à la volée.
    \item[--] Construction de rapports aisée.
    \end{itemize}
\item[A.4.] OLTP  :
    \begin{itemize}
    \item[--] Modification et intégration de données et métadonnées en temps réel.
    \item[--] Volumétries de données
    \item[--] Variété des données
    \end{itemize}
\end{itemize}
\begin{figure}[!ht]\centering
\includegraphics[scale=0.5]{"example Shuttle grid".png}
\caption{Exemple de grille Shuttle}
\label{fig:fig1}
\end{figure}
\FloatBarrier
