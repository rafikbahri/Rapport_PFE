\chapter*{Résumé}
\addcontentsline{toc}{chapter}{Résumé}
%===================================================================

Ce rapport couvre les phases de conception, développement, test et mise en place des chaines d'intégration et livraison continues, et les travaux d'automatisation effectués au sein de la société Konvergence. Nous expliquerons aussi les étapes de chaque phase en justifiant les choix technologiques ainsi que les paradigmes utilisés.\\ 

Nous allons commencer par mettre en place trois chaines d'intégration et livraison continues, pour les différentes versions de l'application IFRS16, un produit principal basé sur la plateforme Shuttle, qu'on découvrira dans le deuxième chapitre. Ces chaines permettent la construction de l'application IFRS16, le test et le retour rapide des éventuelles anomalies.\\

La deuxième partie du stage consiste à concevoir un mécanisme automatique de lancement des tests de performance pour la plateforme Shuttle et de reporting des résultats. Étant donné les grosses volumétries gérées par les applications Shuttle, la stabilité des temps de réponse est une contrainte très importante.\\

Ces projets ont nécessité tout d’abord une familiarisation avec la solution Shuttle, une solution logicielle unifiée et collaborative dédiée au pilotage de la décision dans les domaines de la performance financière et opérationnelle et éditée et maintenue par Konvergence business \& Technologies, ainsi que les outils utilisés pour le déploiement et test de la plateforme. Une étude des concepts de base de l’intégration continue, les tests automatiques et la livraison continue s'impose pour bien acquérir les connaissances nécessaires afin d'implémenter un système adapté à un produit complexe tel que Shuttle.\\