\setcounter{figure}{0} 
\setcounter{table}{0}
\setcounter{footnote}{0}
\setcounter{equation}{0}
\pagestyle{fancy}
\fancyhf{}
\renewcommand{\chaptermark}[1]{\markboth{\MakeUppercase{#1 }}{}}
\renewcommand{\sectionmark}[1]{\markright{\thesection~ #1}}
\fancyhead[RO]{\bfseries\rightmark}
\fancyhead[LE]{\bfseries\leftmark}
\fancyfoot[RO]{\thepage}
\fancyfoot[LE]{\thepage}
\renewcommand{\headrulewidth}{0.5pt}
\renewcommand{\footrulewidth}{0pt}

\makeatletter
\renewcommand\thefigure{B.\arabic{figure}}
\renewcommand\thetable{B.\arabic{table}} 
\makeatother

\chapter{Annexe B : L'application K-IFRS16 }
\graphicspath{{Annexe1/figures/}}
%==========================================================================

%    Annexe

%===========================================================================
Le "Manifeste Agile" est une déclaration formelle, qui présente le fruit d’une réunion qui a eu lieu aux États-Unis en 2001 rassemblant 17 experts du développement logiciel. L’objet de cette réunion était d’élaborer une méthode unificatrice, afin de tenir les délais d’un projet tout en tenant également le budget.

Cette déclaration est composée de quatre valeurs fondamentales et de 12 principes qui allaient guider les réflexions futures sur la planification et la gestion d’un projet pour une approche du développement logiciel itérative et centrée sur les personnes. Les 12 principes énoncés par le manifeste ont été adaptés à la gestion de nombreux projets, notamment à l'informatique décisionnelle (BI). Ces principes sont les suivants :

\begin{enumerate}
    \item Satisfaire le client en livrant au plus tôt et de manière constante des fonctionnalités à grande valeur ajoutée.
    \item Accueillir les demandes d changement, même tard dans le développement. Les méthodes Agiles exploitent le changement pour donner un avantage compétitif du client.
    \item Livrer régulièrement logiciel qui fonctionne, toutes les 2 semaines à 2 mois, avec une préférence pour les fréquences les plus rapides.
    \item Maitrise d'ouvrage et développeurs doivent collaborer quotidiennement pendant tout le projet.
    \item Bâtir le projet avec des personnes motivées. Leur donner l'environnement et le support dont elles ont besoin et croire en leur capacité à accomplir le travail.
    \item La méthode la plus efficace et effective de transmettre l'information au sein et à destination d'une équipe est le face à face.
    \item Le logiciel qui marche est la principale mesure de progrès.
    \item Les méthodes Agiles favorisent un rythme de développement soutenable.
    \item Maintenir un rythme de développement constant.
    \item Simplicité, l’art de maximiser la quantité de travail non fait, est essentielle.
    \item Les meilleures architectures, spécifications des besoins et conceptions sont issues d'équipes auto-organisées.
    \item À intervalles réguliers, l'équipe réfléchit aux moyens de devenir plus efficace, puis règle et ajuste son comportement dans ce sens.
\end{enumerate}
Les 4 ces principales valeurs agiles sont les suivantes : 
\begin{itemize}
    \item Les individus et leurs interactions plus que les processus et les outils.
    \item Un logiciel qui fonctionne plus qu’une documentation exhaustive.
    \item La collaboration avec les clients plus que la négociation contractuelle.
    \item L’adaptation au changement plus que le suivi d’un plan.

\end{itemize}
