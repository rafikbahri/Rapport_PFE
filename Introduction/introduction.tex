\chapter*{Introduction Générale}
\addcontentsline{toc}{chapter}{Introduction Générale}
\begin{spacing}{1.2}
%==================================================================================================%

L’origine du terme “Intégration Continue” revient à l’américain Grady Booch\cite{gardBooch}, co-développeur du langage de modélisation UML, qui a introduit le terme ‘Continuous Integration'\cite{continuousIntegration} dans sa méthode de développement de logiciels orientés objets. En revanche, Booch n’a pas préconisé le terme. L’intégration continue revient dans le monde de développement logiciel en mai 2006 avec le Britannique Martin Fowler, un des gros acteurs dans le domaine de conception de logiciels d’entreprise. 
Fowler a expliqué et mit l’emphase sur le concept dans sa publication\cite{martinFowler}.\\

“Continuous Integration is a software development practice where members of a team integrate their work frequently, usually each person integrates at least daily - leading to multiple integrations per day. Each integration is verified by an automated build (including test) to detect integration errors as quickly as possible. Many teams find that this approach leads to significantly reduced integration problems and allows a team to develop cohesive software more rapidly.” [Continuous Integration, Martin Fowler, 1 Mai 2006]\\

Offrant une large gamme d'applications d'aide à la décision dans différents domaines, Konvergence Business \& Technologies s'engage pour satisfaire ses clients, avec une suite logicielle de qualité, testée et livrée dans les temps, avec une ouverture pour les retours d'expériences et le signalement des disfonctionnements (bugs).\\

C'est dans cette perspective, et pour assurer un processus d'intégration, de test et de livraison continues de ses applications, le département recherche et développement de la boite a entamé la mise au point de la pratique DevOps qui part des besoins, bugs, et retours clientes, à la livraison des applications.\\

Dans ce que suit, nous présentons la structure de notre rapport:
\begin{itemize}
\setlength\itemsep{0em}
\item Le premier chapitre intitulé "Cadre du projet" dans lequel, nous présenterons la société d’accueil, la méthodologie de travail et quelques concepts de base.
\item Le deuxième chapitre "IFRS16 CI/CD Pipelines" présentera les chaines d'intégration continue et l'automatisation de tests de l'application IFRS16.
\item Le troisième chapitre " ench Tests Pipeline" présentera la conception et la mise en place du pipeline de Bench Tests, les tests de performance de la plateforme Shuttle. 
\end{itemize}

Nous finissons notre rapport par une conclusion générale dans laquelle nous discuterons les différents travaux réalisés et les perspectives.
\end{spacing}