\setcounter{figure}{0} 
\setcounter{table}{0}
\setcounter{footnote}{0}
\setcounter{equation}{0}
\pagestyle{fancy}
\fancyhf{}
\renewcommand{\chaptermark}[1]{\markboth{\MakeUppercase{#1 }}{}}
\renewcommand{\sectionmark}[1]{\markright{\thesection~ #1}}
\fancyhead[RO]{\bfseries\rightmark}
\fancyhead[LE]{\bfseries\leftmark}
\fancyfoot[RO]{\thepage}
\fancyfoot[LE]{\thepage}
\renewcommand{\headrulewidth}{0.5pt}
\renewcommand{\footrulewidth}{0pt}

\makeatletter
\renewcommand\thefigure{E.\arabic{figure}}
\renewcommand\thetable{E.\arabic{table}} 
\makeatother

\chapter{Annexe E : IFRS 16 }
\graphicspath{{Annexe1/figures/}}
%==========================================================================

%    Annexe

%===========================================================================
Le monde de la comptabilité se préparer pour accueillir une nouvelle norme, IFRS 16, de comptabilisation des contrats de location qui entrera en vigueur le 1er janvier 2019. Elle a pour objectif d'assurer plus de transparence au niveau de la comparabilité entre les sociétés, indépendamment de leurs modes de financements, à travers l'obligation de porter au bilan tous les contrats de location qui présentent un engagement financier qui pèse sur l'entreprise dont la durée est inférieure ou égale à un an et les contrats portants sur des biens de faible valeur (5.000 USD de la valeur du bien à neuf, à titre indicatif), ce qui fait la différence avec la norme précédente (IAS 17)